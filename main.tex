\documentclass[12pt,a4paper]{article}
\usepackage[utf8]{inputenc}
\renewcommand{\baselinestretch}{1.2}
\title{Ranking King Saud University Computer Science Curriculum}
\author{Mohand Alrasheed}
\date{2022 / January / 2}

\begin{document}

\maketitle
\clearpage

\tableofcontents
\clearpage

% \paragraph{License}
% \paragraph{Main developer}
% \paragraph{Initial release date}
% \paragraph{Supported programming languages}
% \paragraph{Supported operating systems}
% \paragraph{High-level features}
% \paragraph{Does it offer a graphical user interface (GUI)?}
% \paragraph{Common uses}
% \paragraph{Advantages and disadvantages}




\section{Introduction}
King Saud University's computer science curriculum has many pros and cons, in this document we outline a criteria for judging courses and ranking the curriculum based on experts' opinions.

\section{Assumptions} 
We make a number of assumptions regarding the evaluation of courses, these guide the evaluation process so they should be read carefully before moving on to results.

\paragraph{Assumptions}
\begin{enumerate}
    \item The course instructor's influence is minimized when ranking courses by imagining all courses being taught by a perfect instructor.
    \item The ranking has been chosen by students mostly on their final semester, which would mean most of them have not studied the last semester and in turn we will not take its courses into consideration in this version.
    \item The ranking has been chosen by students mostly belonging batch 439 and 438, different batches might rank courses differently as course content changes slowly.
    \item Most electives are not taken into consideration, but a small number of popular electives will be considered. 
\end{enumerate}

\section{Criteria}
Courses are judged based on five criteria chosen by the most elite members of 439.

\paragraph{Criteria}
\begin{enumerate}
    \item \textbf{Applications}: This refers to real-world applications of the knowledge gained by studying the course.
    \item \textbf{Relevance}: This refers to the new-ness of the knowledge taught compared to the current (as of this document's date) state of the art.
    \item \textbf{Insight}: This refers to the quality of the knowledge gained with respect to understanding the world and expanding one's horizons.
    \item \textbf{Understanding}: This refers to the proportion of the course's understanding portions over the memorization portions. 
    % \item \textbf{Completeness}: This refers to how connected the course is as a package. A course with randomly assembled useful info will score badly on this category, while a cohesively assembled useless info scores highly.
    \item \textbf{Ease}: This refers to how easy the course was.
\end{enumerate}

\section{Method}
\subsection{Data collection \& processing}
The data is collected and sorted using \emph{Google Forms} in which each reviewer identifies their university batch and sex as well as rate an optional number of courses by an optional number of criteria, i.e. the reviewer can only review one metric or all metrics as well as one courses or all courses. Each reviewer can optionally also provide extra notes alongside their review.

After collecting the data we average each review scores by criteria and course, ending up with courses X Criteria matrix.

\subsection{Score calculation}
After collecting and processing the data we will rank courses using a weighted mean of criteria, we will use three different weights.

\paragraph{Weighing technique}
\begin{enumerate}
    \item \textbf{General}: This weighing takes everything into account fairly, equivalent to a traditional mean. \[Applications\; +\; Relevance\; +\; Insight\; +\; Understanding\; +\; Ease\]
    \item \textbf{Real-world}: This score mainly focuses on real world utility. \[Applications * 1.5\; +\; Relevance\; +\; Insight * 0\; +\; Understanding * 0.2\; +\; Ease * 0.5\]
    \item \textbf{Academic}: This score only focuses on the academic aspect of courses. \[Applications * 0\; +\; Relevance * 0\; +\; Insight * 1.5\; +\; Understanding * 1.5\; +\; Ease * 0\]
\end{enumerate}


\section{Results}

3 tables with different weights and and notes in google forms
\end{document}
